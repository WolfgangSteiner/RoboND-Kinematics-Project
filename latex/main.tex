%%%%%%%%%%%%%%%%%%%%%%%%%%%%%%%%%%%%%%%%%
% Journal Article
% LaTeX Template
% Version 1.4 (15/5/16)
%
% This template has been downloaded from:
% http://www.LaTeXTemplates.com
%
% Original author:
% Frits Wenneker (http://www.howtotex.com) with extensive modifications by
% Vel (vel@LaTeXTemplates.com)
%
% License:
% CC BY-NC-SA 3.0 (http://creativecommons.org/licenses/by-nc-sa/3.0/)
%
%%%%%%%%%%%%%%%%%%%%%%%%%%%%%%%%%%%%%%%%%

%----------------------------------------------------------------------------------------
%	PACKAGES AND OTHER DOCUMENT CONFIGURATIONS
%----------------------------------------------------------------------------------------

\documentclass[twoside]{article}

%\usepackage[sc]{mathpazo} % Use the Palatino font
%\usepackage[UTF8]{fontenc} % Use 8-bit encoding that has 256 glyphs
\linespread{1.05} % Line spacing - Palatino needs more space between lines
\usepackage{microtype} % Slightly tweak font spacing for aesthetics
\usepackage[english]{babel} % Language hyphenation and typographical rules
\usepackage{amsmath}

\usepackage[hmarginratio=1:1,top=1in,left=1in,columnsep=20pt]{geometry} % Document margins
\usepackage[hang, small,labelfont=bf,up,textfont=it,up]{caption} % Custom captions under/above floats in tables or figures
\usepackage{booktabs} % Horizontal rules in tables

\usepackage{lettrine} % The lettrine is the first enlarged letter at the beginning of the text

\usepackage{enumitem} % Customized lists
\setlist[itemize]{noitemsep} % Make itemize lists more compact

\usepackage{abstract} % Allows abstract customization
\renewcommand{\abstractnamefont}{\normalfont\bfseries} % Set the "Abstract" text to bold
\renewcommand{\abstracttextfont}{\normalfont\small\itshape} % Set the abstract itself to small italic text

\usepackage{titlesec} % Allows customization of titles
\renewcommand\thesection{\Roman{section}} % Roman numerals for the sections
\renewcommand\thesubsection{\roman{subsection}} % roman numerals for subsections
\titleformat{\section}[block]{\large\scshape\centering}{\thesection.}{1em}{} % Change the look of the section titles
\titleformat{\subsection}[block]{\large}{\thesubsection.}{1em}{} % Change the look of the section titles

% \usepackage{fancyhdr} % Headers and footers
% \pagestyle{fancy} % All pages have headers and footers
% \fancyhead{} % Blank out the default header
% \fancyfoot{} % Blank out the default footer
% %\fancyhead[C]{Running title $\bullet$ May 2016 $\bullet$ Vol. XXI, No. 1} % Custom header text
% \fancyfoot[RO,LE]{\thepage} % Custom footer text

\usepackage{titling} % Customizing the title section
\usepackage{hyperref} % For hyperlinks in the PDF
\usepackage{flushend}
\usepackage{tabularx}
\usepackage{listings}
\usepackage{xcolor}
\usepackage{graphicx}


\renewcommand{\c}{\text{c}}
\newcommand{\s}{\text{s}}
\newcommand{\pihalf}{\frac{\pi}{2}}
\newcommand{\T}[2]{\mbox{$_{#2}^{#1}{T}$}}
\newcommand{\R}[2]{\mbox{$_{#2}^{#1}{R}$}}
\newcommand{\code}[1]{\texttt{#1}}
\newcommand{\acos}{\text{acos}}
\newcommand{\asin}{\text{asin}}
\newcommand{\figref}[1]{Fig.~\ref{fig:#1}}
\newcommand{\tabref}[1]{Tab.~\ref{fig:#1}}

\lstset{
  %frame=tb,
  language=Python,
  aboveskip=3mm,
  belowskip=3mm,
  showstringspaces=false,
  columns=flexible,
  basicstyle={\small\ttfamily},
  numbers=none,
  % numberstyle=\tiny\color{gray},
  % keywordstyle=\color{blue},
  % commentstyle=\color{dkgreen},
  % stringstyle=\color{mauve},
  breaklines=true,
  breakatwhitespace=true,
  tabsize=3
}
%----------------------------------------------------------------------------------------
%	TITLE SECTION
%----------------------------------------------------------------------------------------

\setlength{\droptitle}{-4\baselineskip} % Move the title up

\pretitle{\begin{center}\Huge\bfseries} % Article title formatting
\posttitle{\end{center}} % Article title closing formatting
\title{Pick and Place Project Report} % Article title
\author{%
\textsc{Wolfgang Steiner} \\[0.5ex] % Your name
%\normalsize University of California \\ % Your institution
\normalsize \href{mailto:wolfgang.steiner@gmail.com}{wolfgang.steiner@gmail.com} % Your email address
%\and % Uncomment if 2 authors are required, duplicate these 4 lines if more
%\textsc{Jane Smith}\thanks{Corresponding author} \\[1ex] % Second author's name
%\normalsize University of Utah \\ % Second author's institution
%\normalsize \href{mailto:jane@smith.com}{jane@smith.com} % Second author's email address
}
\date{\today} % Leave empty to omit a date
\renewcommand{\maketitlehookd}{%
% \begin{abstract}
% \noindent
% \end{abstract}
}

%----------------------------------------------------------------------------------------

\begin{document}

% Print the title
\maketitle

%----------------------------------------------------------------------------------------
%	ARTICLE CONTENTS
%----------------------------------------------------------------------------------------

\section{DH Parameters and Forward Kinematics}
\begin{figure}[ht]
  \centering \input{kuka_arm.pdf_tex}
  \caption{Schematic of the Kuka KR210 arm.}
  \label{fig:schematic}
\end{figure}


\begin{table}[ht]
\caption{Modified DH parameters of the Kuka KR210 arm.}
\label{tab:dh-parameters}
%\setlength{\tabcolsep}{1pt}
%\scriptsize
\centering
\begin{tabular}{l|rrrc}
        & $\alpha_{i-1}$ & $a_{i-1}$ & $d_i$ & $\theta_i$ \\
\hline
$\T{0}{1}$ &                0 &    0   & 0.750 & $\theta_1$ \\
$\T{1}{2}$ & $-\frac{\pi}{2}$ &  0.350 &     0 & $\theta_2 - \frac{\pi}{2}$ \\
$\T{2}{3}$ &                0 &  1.250 &     0 & $\theta_3$ \\
$\T{3}{4}$ & $-\frac{\pi}{2}$ & -0.054 & 1.500 & $\theta_4$ \\
$\T{4}{5}$ & $ \frac{\pi}{2}$ &      0 &     0 & $\theta_5$ \\
$\T{5}{6}$ & $-\frac{\pi}{2}$ &      0 &     0 & $\theta_6$ \\
$\T{6}{G}$ &  $           0 $ &      0 & 0.303 & $0$ \\
\end{tabular}

\end{table}

\begin{equation}
\T{i-1}{i} =
\begin{bmatrix}
  \c\theta_i & -\s\theta_i & 0 & a_{i-1} \\
  \s\theta_i\c\alpha_{i-1} & \c\theta_i\c\alpha_{i-i} & -\s\alpha_{i-1} & -\s\alpha_{i-1}d_i \\
  \s\theta_i\s\alpha_{i-1} & \c\theta_i\s\alpha_{i-i} &  \c\alpha_{i-1} &  \c\alpha_{i-1}d_i \\
  0 & 0 & 0 & 1\\
\end{bmatrix}
\end{equation}

\noindent
\begin{minipage}{0.33\linewidth}
\begin{equation}
\T{0}{1} =
  \begin{bmatrix}
    \c\theta_1 & -\s\theta_1 & 0 & 0 \\
    \s\theta_1 &  \c\theta_1 & 0 & 0 \\
            0 &          0 & 1 & d_1 \\
            0 &          0 & 0 & 1 \\
  \end{bmatrix}
\end{equation}
\end{minipage}%
\begin{minipage}{0.67\linewidth}
\begin{equation}
  \T{1}{2} =
  \begin{bmatrix}
  \c(\theta_2 - \pihalf) & -\s(\theta_2 - \pihalf) &  0 & a_1 \\
                       0 &                       0 &  1 &   0 \\
  -\s(\theta_2 - \pihalf)&  \c(\theta_2 - \pihalf) &  0 &   0 \\
                       0 &                       0 &  0 &   1 \\
  \end{bmatrix}
  =
  \begin{bmatrix}
  \s\theta_2 &  \c\theta_2 & 0 & a_1 \\
           0 &           0 & 1 &   0 \\
  \c\theta_2 &  \s\theta_2 & 0 &   0 \\
           0 &           0 & 0 &   1 \\
  \end{bmatrix}
\end{equation}
\end{minipage}

\vspace{0.5ex}

\noindent\begin{minipage}{0.33\linewidth}
\begin{equation}
  \T{2}{3} =
  \begin{bmatrix}
    \c\theta_3 & -\s\theta_3 & 0 & a_2 \\
    \s\theta_3 &  \c\theta_3 & 0 &   0 \\
             0 &           0 & 1 &   0 \\
             0 &           0 & 0 &   1 \\
  \end{bmatrix}
\end{equation}
\end{minipage}%
\begin{minipage}{0.33\linewidth}
\begin{equation}
\T{3}{4} =
\begin{bmatrix}
  \c\theta_4 & -s\theta_4 & 0 & a_3 \\
           0 &          0 & 1 & d_4 \\
 -\s\theta_4 & -c\theta_4 & 0 &   0 \\
           0 &           0 & 0 &  1 \\
\end{bmatrix}
\end{equation}
\end{minipage}
\begin{minipage}{0.33\linewidth}
\begin{equation}
\T{4}{5} =
\begin{bmatrix}
  \c\theta_5 & -\s\theta_5 &  0 & 0 \\
           0 &           0 & -1 & 0 \\
  \s\theta_5 &  \c\theta_5 &  0 & 0 \\
           0 &           0 &  0 & 1 \\
\end{bmatrix}
\end{equation}
\end{minipage}%

\begin{minipage}{0.33\linewidth}
\begin{equation}
\T{5}{6} =
\begin{bmatrix}
  \c\theta_6 & -\s\theta_6 & 0 & 0 \\
           0 &           0 & 1 & 0 \\
 -\s\theta_6 & -\c\theta_6 & 0 & 0 \\
           0 &           0 & 0 & 1 \\
\end{bmatrix}
\end{equation}
\end{minipage}%
\begin{minipage}{0.33\linewidth}
\begin{equation}
\T{6}{G} =
\begin{bmatrix}
  1 & 0 & 0 &   0 \\
  0 & 1 & 0 &   0 \\
  0 & 0 & 1 & d_G \\
  0 & 0 & 0 &   1 \\
\end{bmatrix}
\end{equation}
\end{minipage}%
\begin{minipage}{0.33\linewidth}
\end{minipage}

\section{Calculation of the Wrist Center Position}
The orientation of the gripper is supplied to the inverse kinematic server as a quaternion.
From this, the rotation matrix relative to the base frame can be computed using the function
\code{tf.transformations.quaternion\_matrix}, which is equivalent to the following:
\begin{equation}
  R_{gripper} =
  \begin{bmatrix}
    1 - 2 q_y^2 - 2 q_z^2	& 2 q_x q_y - 2 q_z q_w	& 2 q_x q_z + 2 q_y q_w \\
    2 q_x q_y + 2 q_z q_w	& 1 - 2 q_x^2 - 2 q_z^2	& 2 q_y q_z - 2 q_x q_w \\
    2 q_x q_z - 2 q_y q_w	& 2 q_y q_z + 2 q_x q_w	& 1 - 2 q_x^2 - 2 q_y^2
  \end{bmatrix}
  \cdot R_{corr}
\end{equation}

Alternatively, the roll ($\phi$), pitch ($\theta$), and yaw ($\psi$) angles can be computed with the function
\\ \code{transformations.euler\_from\_quaternion}, by specifiying the \code{'rzyx'} option.
This computes the Euler angles relative to the local frame in the order of a rotation about the
z, y and x axes in this order. With these Euler angles, the total rotation matrix can be computed as:

\begin{equation}
  R_{gripper} = R_z(\psi) R_y(\theta) R_x(\phi) R_{corr}
=
  \begin{bmatrix}
    \c\theta \c\psi& \s\phi \s\theta \c\psi - \s\psi \c\phi&  \s\phi \s\psi + \s\theta \c\phi \c\psi \\
    \s\psi \c\theta& \s\phi \s\theta \s\psi + \c\phi \c\psi& -\s\phi \c\psi + \s\theta \s\psi \c\phi \\
          -\s\theta&                        \s\phi \c\theta&                         \c\phi \c\theta
  \end{bmatrix}
  \cdot R_{corr}
\end{equation}

With this rotation matrix, the total transformation matrix from the base frame to the gripper
frame can be expressed as:
\begin{equation}
  T_{gripper} =
  \begin{bmatrix}
     \s\phi\s\psi + \s\theta\c\phi\c\psi & -\s\phi\s\theta\c\psi + \s\psi\c\phi & \c\theta\c\psi & p_x \\
    -\s\phi\c\psi + \s\theta\s\psi\c\phi & -\s\phi\s\theta\s\psi - \c\phi\c\psi & \s\psi\c\theta & p_y \\
     \c\phi\c\theta                      & -\s\phi\c\theta                      & -\s\theta      & p_z \\
                                       0 &                                    0 &              0 &   1
  \end{bmatrix}
\end{equation}

The third column of this transformation matrix represents the z-axis of the griper frame $N_{z,gripper}$.
XXXXXXX

\begin{equation}
  \begin{bmatrix}
    w_x \\ w_y \\ w_z
  \end{bmatrix}
  =
  \begin{bmatrix}
    p_x \\ p_y \\ p_z
  \end{bmatrix}
  -
  \begin{bmatrix}
    \c\theta\c\psi \\ \s\psi\c\theta \\ -\s\theta
  \end{bmatrix}
  \cdot d_G
\end{equation}


\section{Inverse Kinematics of the Wrist Center}
The next step in the inverse kinematic problem lies in determining the angles
$\theta_1$, $\theta_2$, and $\theta_3$ in such a way that the wrist center of the Kuka arm
lies at the coordinates $w_x$, $w_y$, and $w_z$. To achieve this, the angle $\theta_1$
can easily be computed in the xy-plane of the base frame from:
\begin{equation}
  \theta_1 = \text{atan2}(w_y, w_x)
\end{equation}

After computing $\theta_1$, the problem of solving for $\theta_2$ and $\theta_3$ can be
transformed into a second 2D problem in the xz-plane of the Kuka arm, as shown in \figref{sideview}.
For this, the wrist center is transformed into the reference frame of link 1:
\begin{figure}[ht]
  \centering\input{sideview.pdf_tex}
  \caption{Sideview of the Kuka arm.}
  \label{fig:sideview}
\end{figure}

\begin{equation}
\begin{matrix}
  w_{tx} & = & w_x \c\theta_1 + w_y \s\theta_1 & &\\
  w_{ty} & = & - w_x  \s\theta_1 + w_y \c\theta_1 & = &0\\
  w_{tz} & = & w_z - d_1&& \\
\end{matrix}
\end{equation}

\begin{equation}
\T{4}{1} =
\begin{bmatrix}
  \s(\theta_2 + \theta_3)\c\theta_4 & -\s\theta_4\s(\theta_2 + \theta_3) & \c(\theta_2 + \theta_3) & a_1 + a_2\s(\theta_2) + a_3\s(\theta_2 + \theta_3) + d_4\c(\theta_2 + \theta_3) \\
  -\s\theta_4 & -\c\theta_4 & 0 & 0 \\
  \c\theta_4\c(\theta_2 + \theta_3) & -\s\theta_4\c(\theta_2 + \theta_3) & -\s(\theta_2 + \theta_3) & a_2\c(\theta_2) + a_3\c(\theta_2 + \theta_3) - d_4\s(\theta_2 + \theta_3) \\
  0 & 0 & 0 & 1\\
  \label{eq:trans14}
\end{bmatrix}
% =
% \begin{bmatrix}
% & R_1^4& & \begin{matrix} w_{tx} \\ w_{ty} \\ w_{tz} \end{matrix} \\
% 0 & 0 & 0 & 1 \\
% \end{bmatrix}
\end{equation}

The entries of the rightmost column of the matrix $\T{4}{1}$ represent the translation of the
wrist center relative to the origin of link 1, so we can extract the following two equations from this:

\begin{eqnarray}
   w_{tx} - a_1 &=& a_2\s(\theta_2) + a_3\s(\theta_2 + \theta_3) + d_4\c(\theta_2 + \theta_3) \\
   w_{tz}       &=& a_2\c(\theta_2) + a_3\c(\theta_2 + \theta_3) - d_4\s(\theta_2 + \theta_3)
\end{eqnarray}

The angle $\theta_3$ effectively defines the reach or radius of the robot arm, which has to be equal
to the distance between link one and the transformed wrist center position in the plane of the arm.
Thus by computing the squared distance we get:

\begin{equation}
\begin{split}
(w_{tx} - a_1)^2 + w_{tz}^2
  % &= a_2^2 (\s^2\theta_2 + \c^2\theta_2) + a_3^2 (\s^2(\theta_2 + \theta_3) + \c^2(\theta_2 + \theta_3)) + d_4^2(\s^2(\theta_2 + \theta_3) + \c^2(\theta_2 + \theta_3)) \\
  % &+ 2 a_2 a_3 \s\theta_2\s(\theta_2 + \theta_3) + 2 a_2 d_4 \s\theta_2\c(\theta_2 + \theta_3) + 2 a_3 d_4 \s(\theta_2+\theta_3)\c(\theta_2 + \theta_3) \\
  % &+ 2 a_2 a_3 \c\theta_2\c(\theta_2 + \theta_3) - 2 a_2 d_4 \c\theta_2\s(\theta_2 + \theta_3) - 2 a_3 d_4 \c(\theta_2+\theta_3)\s(\theta_2 + \theta_3) \\
  = a_2^2 + a_3^2 + d_4^2 + 2 a_2 a_3 \c\theta_3 - 2 a_2 d_4\s\theta_3 \\
\end{split}
\end{equation}
\begin{equation}
  \frac{\rho^2 - \sigma^2}{2 a_2} = a_3\c\theta_3 - d_4\s\theta_3 = \sqrt{a_3^2 + d_4^2} \cdot \sin(\theta_3 + \text{atan2}(a_3, -d_4))
  \label{eq:theta3}
\end{equation}
\begin{equation}
  \rho^2 = (w_{tx} - a_1)^2 + w_{tz}^2
\end{equation}
\begin{equation}
  \sigma^2 = a_2^2 + a_3^2 + d_4^2
\end{equation}

% \begin{equation}
%   \alpha\sin(x) + \beta\cos(x) = \gamma \sin(x + \phi), \ \text{with:}\ \gamma = \sqrt{\alpha^2 + \beta^2},\ \phi = \text{atan2}(\beta, \alpha)
% \end{equation}

Solving \eqref{eq:theta3} for $\theta_3$ finally yields:
\begin{equation}
  \theta_3 = \text{asin}\left(\frac{\rho^2 - \sigma^2}{2 a_2 \sqrt{a_3^2 + d_4^2}}\right) - \text{atan2}(a_3, -d_4)
\end{equation}

With $\theta_3$ determined, it now becomes possible to solve for $\theta_2$.
This sub-problem can be further divided into the two angles $\theta_{21}$ and $\theta_{22}$ (cf. Fig. 2).
The angle $\theta_{21}$ can be thought of as the angle required to place the tip of a straight arm
with length $\rho$ at the wrist center. It be can easily computed as:
\begin{equation}
  \theta_{21} = \text{atan2}(w_{tz}, w_{tx} - a_1)
\end{equation}

The angle $\theta_{22}$ can now be solved by utilizing the matrix element $\T{1}{4}_{33}$ in
\eqref{eq:trans14}. Here, the vertical displacement $w_{tz}$
has already been accounted for by $\theta_{21}$ so that this matrix element must be equal to zero for $\theta_{22}$:
\begin{equation}
  a_2\c\theta_{22} + a_3\c(\theta_{22} + \theta_3) - d_4\s(\theta_{22} + \theta_3)
    = \alpha_{22} \sin(\theta_{22} + \phi_{22}) = 0
\end{equation}

Solving this equation for $\theta_{22}$ yields:
\begin{equation}
  \theta_{22} = -\text{atan2}(-a_3\s\theta_3 - d_4\c\theta_3, a_2 + a_3\c\theta_3 - d_4\s\theta_3) + n\pi
\end{equation}

Finally, the two angles $\theta_{21}$ and $\theta_{22}$ can be combined:
\begin{equation}
  \theta_2 = \frac{\pi}{2} - \theta_{21} - \theta_{22}
\end{equation}

\section{Inverse Kinematics of the Gripper Orientation}
Once the first three joint angles have been determined, the inverse kinematics problem can be
completed by computing the orientation of the tool relative to the wrist center. The desired
orientation of the gripper is supplied to the IK solver in form of a quaternion \ref{quaternion}.
As the angles $\theta_1$, $\theta_2$ and $\theta_3$ are already known from the previous steps,
it is now possible to compute the rotation matrix $\R{3}{6}$ by premultiplying the matrix $R_{gripper}$
with the inverse of the composite rotation $\R{0}{3}$:
\begin{equation}
  \R{3}{6} = (\R{0}{1}\cdot \R{1}{2}\cdot \R{2}{3})^{-1} R_{gripper} =
  \begin{bmatrix}
    -\s\theta_4 \s\theta_6 + \c\theta_4\c\theta_5 \c\theta_6 & -\s\theta_4 \c\theta_6 - \s\theta_6 \c\theta_4 \c\theta_5 & -\s\theta_5 \c\theta_4 \\
                                        \s\theta_5 \c\theta_6 &                                    -\s\theta_5 \s\theta_6 &             \c\theta_5 \\
    -\s\theta_4 \c\theta_5 \c\theta_6 - \s\theta_6 \c\theta_4 &  \s\theta_4 \s\theta_6 \c\theta_5 - \c\theta_4 \c\theta_6 &  \s\theta_4 \s\theta_5
  \end{bmatrix}
\end{equation}

From this matrix, $\theta_5$ can be directly computed as:
\begin{equation}
  \theta_5 = \pm\acos(\R{3}{6}_{23})
\end{equation}

To guarantee that the gripper will point into the correct direction and not its opposite,
it is important to compute $\theta_4$  by satisfying \emph{two} entries of the matrix
$\R{3}{6}$.
\begin{equation}
  \frac{\R{3}{6}_{33}}{-\R{3}{6}_{13}} = \frac{\s\theta_4\s\theta_5}{\c\theta_4\s\theta_5} = \tan(\theta_4)
\end{equation}

From this, $\theta_4$ can be computed as follows. Here, I set $\theta_4$ to zero for very small
$\theta_5$. In this case, the correct orientation of the gripper is achieved by $\theta_6$.
\begin{equation}
  \theta_4
  \begin{cases}
    \text{atan2}\left(\R{3}{6}_{33}, -\R{3}{6}_{13}\right) & \text{if}\ |\theta_5| > \epsilon \\
    0                                           & \text{if}\ |\theta_5| \leq \epsilon
  \end{cases}
\end{equation}

Finally, $\theta_6$ can be computed from one of two cases: If $\theta_5$, and
consequently $\theta_4$ are close to zero, the first entry $\R{3}{6}_{11}$ simplifies to
$\c\theta_6$. Otherwise, $\theta_6$ can easily be computed from $\R{3}{6}_{21}$.
Thus, we get the following expressions for computing $\theta_6$:
\begin{equation}
  \theta_6 =
  \begin{cases}
     \text{acos}\left(\frac{\R{3}{6}_{21}}{\sin(\theta_5)}\right) & \text{if}\ |\theta_5| > \epsilon \\
     \text{acos}\left(\R{3}{6}_{11}\right)                & \text{if}\ |\theta_5| \leq \epsilon
  \end{cases}
\end{equation}

%%%%%%%%%%%%%%%%%%%%%%%%%%%%%%%%%%%%%%%%%%%%%%%%%%%%%%%%%%%%%%%%%%%%%%%%%%%%%%%%%%%%%%%%%%%%
\section{Results}
%%%%%%%%%%%%%%%%%%%%%%%%%%%%%%%%%%%%%%%%%%%%%%%%%%%%%%%%%%%%%%%%%%%%%%%%%%%%%%%%%%%%%%%%%%%%
With the presented inverse kinematics procedure, the robot arm is able to reliably pick up
the object and place it in the receptacle. In order to measure the accuracy of the procedure,
I calculate the root-mean-square error (RMSE) of the gripper position in the following way:
at each time step I perform a forward kinematics pass with the calculated joint angles
in order to compute the predicted gripper position $p_{pred,i}$, which is then compared
to the ground truth position $p_{gt,i}$ (the original gripper coordinates provided by the path planner):
\begin{equation}
  \text{RMSE(p)} = \sqrt{\frac{1}{n}\sum\limits_i(p_{pred,i} - p_{gt,i})^2}
\end{equation}

The resulting RMSE values for three typical pick and place cycles are summarized in
\tabref{rmse}. As can be seen, the RMSE values are in the magnitude of floating point
rounding errors, thus supporting the correctness of the presented inverse kinematics procedure.

\begin{table}[ht]
  \caption{RMSE values of the predicted gripper position for three typical pick and place cycles.}
  \centering\begin{tabular}{c|ccc}
    n & $\text{RMSE}(p_x)/m$ & $\text{RMSE}(p_y)/m$ & $\text{RMSE}(p_z)/m$ \\
    \hline
    1 & $1.38372517 \cdot 10^{−16}$ & $2.50313645 \cdot 10^{−16}$ & $1.01349032 \cdot 10^{−16}$\\
    2 & $8.69068332 \cdot 10^{−15}$ & $1.21391017 \cdot 10^{−10}$ & $4.44089210 \cdot 10^{−16}$\\
    3 & $1.64697269 \cdot 10^{−15}$ & $2.23592115 \cdot 10^{−11}$ & $3.59867566 \cdot 10^{−16}$
  \end{tabular}
\end{table}


%----------------------------------------------------------------------------------------
%	REFERENCE LIST
%----------------------------------------------------------------------------------------

%\bibliography{research_review}
%\bibliographystyle{ieeetr}
%----------------------------------------------------------------------------------------

\end{document}
